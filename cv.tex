%% start of file `template.tex'.
%% Copyright 2006-2013 Xavier Danaux (xdanaux@gmail.com).
%
% This work may be distributed and/or modified under the
% conditions of the LaTeX Project Public License version 1.3c,
% available at http://www.latex-project.org/lppl/.


\documentclass[11pt,a4paper,sans]{moderncv}        % possible options include font size ('10pt', '11pt' and '12pt'), paper size ('a4paper', 'letterpaper', 'a5paper', 'legalpaper', 'executivepaper' and 'landscape') and font family ('sans' and 'roman')

% moderncv themes
\moderncvstyle{classic}                             % style options are 'casual' (default), 'classic', 'oldstyle' and 'banking'
\moderncvcolor{blue}                               % color options 'blue' (default), 'orange', 'green', 'red', 'purple', 'grey' and 'black'
%\renewcommand{\familydefault}{\sfdefault}         % to set the default font; use '\sfdefault' for the default sans serif font, '\rmdefault' for the default roman one, or any tex font name
%\nopagenumbers{}                                  % uncomment to suppress automatic page numbering for CVs longer than one page

% character encoding
%\usepackage[utf8]{inputenc}                       % if you are not using xelatex ou lualatex, replace by the encoding you are using
%\usepackage{CJKutf8}                              % if you need to use CJK to typeset your resume in Chinese, Japanese or Korean

% adjust the page margins
\usepackage[scale=0.75, top=2cm, bottom=2cm]{geometry}
% \usepackage{hyperref}
%\setlength{\hintscolumnwidth}{3cm}                % if you want to change the width of the column with the dates
%\setlength{\makecvtitlenamewidth}{10cm}           % for the 'classic' style, if you want to force the width allocated to your name and avoid line breaks. be careful though, the length is normally calculated to avoid any overlap with your personal info; use this at your own typographical risks...

% personal data
\name{Nikita}{Artyushov}
\address{}{Tartu, Estonia}
\phone[mobile]{+372~5361~5889}
\email{nikita.artyushov@gmail.com}
\social[linkedin][www.linkedin.com/in/artyushov]{Nikita Artyushov}
\social[github]{artyushov}

\begin{document}
\makecvtitle

\section{Experience}

\cventry{Jan~2015--present}{Software Engineer}{\href{http://plumbr.eu}{Plumbr}}{Tartu}{Estonia}{
\textbf{java, groovy, amazon-web-services, spring-boot, docker, gradle, apache-kafka, microservices}
\vspace{1mm}
\newline{}
I played one of the core roles in two major projects at Plumbr.
\vspace{1mm}
\newline{}
The first one is related to memory dumps analysis. I designed and implemented a service which reads heap dump of a java application, builds object graph, analyses it and produces a report about major memory consumers.
\vspace{1mm}
\newline{}
The second project is related to distributed tracing. I was deeply involved into designing the whole concept from the ground up and then developed an application which assembles multiple parts of a distributed transaction into one entity.
\vspace{1mm}
\newline
Besides that there were a lot of other tasks related to various parts of our product, including infrastructure development, front-end, etc.
}

\cventry{July~2011-- Dec~2014}{Software Engineer}{\href{https://yandex.com/company/}{Yandex}}{St. Petersburg}{Russia}{
\textbf{java, spring, mapreduce, bigdata, git, maven}
\vspace{1mm}
\newline
I worked at Search Quality Evaluation department. Our team was developing several internal services used to calculate various search quality metrics. The main pipeline involved preprocessing terabytes of data every day and then using it to efficiently calculate metric values for different search experiments with different filters.
\vspace{1mm}
\newline
My responsibilities included implementation of new metrics designed by data analytics team. I was also in charge of developing and supporting tens of different html parsers which were used for crawling multiple search systems to compare their quality.
}

\section{Education}
\cventry{2008--2013}{Masters}{}{St. Petersburg State University}{}{Faculty of Mathematics and Mechanics, Computer Science Chair}  % arguments 3 to 6 can be left empty

\section{Non-work projects}
\cvitem{}{
\textbf{JMH plugin for Intellij Idea} (\href{https://plugins.jetbrains.com/plugin/7529?pr=idea}{https://plugins.jetbrains.com/plugin/7529?pr=idea}).
\newline{}
The motivation behind this project was to make running JMH benchmarks from IDE as easy as running unit tests. I decided to implement it just to try something new and to contribute both to Intellij Idea and JMH communities.
}

\section{Personal info}
\cvitem{}{I'm strongly motivated by challenging problems like finding the causes of some bizarre bugs or performance bottlenecks as well as writing quality code with good architecture covered with tests. I would be glad to work on a project where the mere knowledge of a programming language won't be enough and where I could learn a lot of stuff from more experienced colleagues. One of the things I couldn't live without is code review.\newline
Besides that, I dance lindy-hop and do other kinds of sports such as: volleyball, football, basketball, whateverball, table tennis, etc.}

\end{document}

%% end of file `template.tex'.